\chapter*{Niveau 3}
\addcontentsline{toc}{chapter}{Niveau 3}

At sælge er at leve, og du har solgt alt mellem himmel og jord. Du kan se, hvordan markedet bevæger sig og dine evner til at sælge og købe gør, at dine priser altid er de bedste, og at dine folk aldrig skal gå sultene i seng. Du har opbygget dit eget lille handelskompagni. 

\begin{table}[H]
    \centering
    \begin{tabular}{|p{0.50\textwidth}|p{0.25\textwidth}|}
    \rowcolor{cerulean!80}\hline
        Evne navn & Pris i XP \\\hline
        Armeret pung & 3\\\hline
        Investorer niv. 3& 2\\\hline
        Markedsanalytiker niv. 1& 2\\\hline
        Markedsanalytiker niv. 2& 3\\\hline
        Vareudvalg niv. 3& 2\\\hline
    \end{tabular}
\end{table}
\section*{Evne beskrivelse}
\addcontentsline{toc}{section}{Evne beskrivelse}

\subsection*{Armeret pung}
\addcontentsline{toc}{subsection}{Armeret pung}
Du kan nu sætte op til tre låse af niveau 3 på, hver af dine egne punge.\\

\subsection*{Investorer Niv. 3}
\addcontentsline{toc}{subsection}{Investorer Niv. 3}
Som handelsmand vil du få udleveret 2d6 fjend pr. niveau ved spilstart.
\\

\subsection*{Markedsanalytiker Niv. 1}
\addcontentsline{toc}{subsection}{Markedsanalytiker Niv. 1}
Varer fra Varerudvalg Niv. 1 vil koste Fjend eller Handelspoint, men ikke begge dele. Denne evne kræver Varerudvalg Niv. 1.\\

\subsection*{Markedsanalytiker Niv. 2}
\addcontentsline{toc}{subsection}{Markedsanalytiker Niv. 2}
Varer fra Varerudvalg Niv. 2 vil koste Fjend eller Handelspoint, men ikke begge dele. Denne evne kræver Varerudvalg Niv. 2.\\

\subsection*{Vareudvalg Niv. 3}
\addcontentsline{toc}{subsection}{Vareudvalg Niv. 3}
Du optjener nu 30 handelspoint pr. senarie. Disse kan du, sammen med Fjend\footnote{Møntfoden i A'kastin}, benytte til at købe vare fra Vareudvalg Niv. 1, 2 og 3. Hvoraf varer fra Niv. 3 er opstillet nedenfor.
\begin{table}[H]
    \centering
    \begin{tabular}{|p{0.50\textwidth}|p{0.25\textwidth}|p{0.25\textwidth}|}
    \hline
    \rowcolor{cerulean!80}
    \multicolumn{3}{c}{Vareudvalg Niv. 3}\\
    \hline
    \rowcolor{cerulean!40}
         Varens navn & Pris i Fjend & Pris i Handelspoint \\\hline
         Dværgerod & 4 & 4\\\hline
         Hybenholt & 4 & 4\\\hline
         Kærlighedsdrik (drik) & 15 & 15\\\hline
         Stjernestøv & 10 & 4 \\\hline
         Lås Niv. 2 & 20 & 16\\\hline
         Sandhedsdrik (drik) & 20 & 15\\\hline
         Slangefrø (urt) & 4 & 4\\\hline
    \end{tabular}
    \end{table}